\documentclass{article} 

\usepackage[utf8]{inputenc} 
\usepackage{amsmath}
\usepackage{physics} 
\usepackage{graphicx}
\usepackage[margin=1.6in]{geometry}
\usepackage[affil-it]{authblk} 
\usepackage{etoolbox} 
\usepackage{lmodern}

\begin{document}

\begin{center}
  \Large{\textbf{
    Popular Summary: \\
    QED theory of electron beam-induced electronic excitation and sputtering
    in 2D crystals
%     QED theory of electronic excitation and sputtering in 2D crystals under
%     electron irradiation
%     and sputtering cross-sections
}}
\end{center}

% context: what do the readers need to know to understand the need?
The ability to control a material's structure atom-by-atom can provide
powerful solutions to many of today’s most pressing technological and
environmental challenges.
To this end, electron beam irradiation can be used for precise material
sculpting by selectively nudging individual atoms with high-speed electrons.
In this way, atomic-scale features can be chiseled into the material through a
process called sputtering, in which a high-speed beam electron knocks an atom
out of the material.

The theory of electron collisions with matter was first developed over 100
years ago, and has since undergone only slight modifications to account for
relativistic and thermal effects.
% need: why is the state of the art not good enough?
However, recent experiments have shown that modern theoretical models often
drastically underestimate sputtering rates in two-dimensional (2D) insulators,
a class of materials possessing enormous potential for nanotechnology.
This is because the majority of current models focus solely on the interaction
between the beam electron and material's nuclei, neglecting the possibility that
the beam electron can also collide with and excite the material's electrons.
In insulators, these electronic excitations can weaken the bonding between the
irradiated atoms, increasing their sputtering rates.

% task / document: what have we/this document done to address the need?
% Using quantum electrodynamics (QED), the most fundamental theory of
% electromagnetism, this paper develops a first-principles method to calculate
% the probability of beam-induced electronic excitations.
To account for this phenomenon, we have developed a first-principles method to
calculate the probability of beam-induced electronic excitations using quantum
electrodynamics (QED), the most fundamental theory of electromagnetism.
We then provide a formalism to compute the degree to which these excitations
can increase a material's sputtering rates.
% Findings: what did we find from carrying out the task?
Lastly, we show that our new model can accurately predict the sputtering rates
in two highly-relevant 2D insulators: hexagonal boron nitride (hBN) and
molybdenum disulfide (MoS$_2$).

% Conclusions: why are the findings important?
This new predictive power can help arm materials engineers with precise
atomic-scale control of any 2D material.
% Perspective: where do we go from here?
We also hope that our QED approach can inspire further exploration into the
applications of electron beams and the physics of beam-matter interactions.

\end{document}

% % context: what do the readers need to know to understand the need?
% The ability to control a material's structure atom-by-atom can provide
% powerful solutions to many of today’s most pressing technological and
% environmental challenges.
% To this end, electron beams are effective tools 
% for engineer the morphology
% of a material by nudging individual atoms with high-speed
% electrons.
% % 2-dimensional (2D)
% These beam-induced structural modifications can be chiseled into the material
% through a process called sputtering, in which a high-speed beam electron knocks
% an atom out of the material.
% It follows that many computational models have been developed to predict the
% rates of sputtering in irradiated materials.
% % need: why is the state of the art not good enough?
% % while these models give reasonable predictions for conductors, they
% % often
% However, these models often drastically underestimate the sputtering rates in
% insulators.
% This is because the majority of them focus solely on interactions between the
% beam electrons and target nuclei, neglecting the possibility of beam-induced
% electronic excitations in the material.
% In insulators, these excitations can weaken the bonds holding the irradiated
% atoms in place, increasing their sputtering rates.
% 
% % task / document: what have we/this document done to address the need?
% To address this phenomenon, this paper develops a first-principles method to
% calculate the probability of beam-induced electronic excitations using quantum
% electrodynamics.
% The paper then demonstrates how these excitations explicitly considers the
% effect of these electronic excitations on sputtering kinetics.
% % Findings: what did we find from carrying out the task?
% Applying this method to 2D hexagonal boron nitride (hBN) and molybdenum
% disulfide MoS$_2$ significantly increases their predicted sputtering rates,
% reducing the disparity between theory and experiment.
% 
% % Conclusions: why are the findings important?
% The model provided here can be easily extended to predict a variety of
% atomic displacement processes in any crystalline material.
% This new predictive power can help arm materials engineers with precise
% atomic-scale control of any 2D material.
% % Perspective: where do we go from here?
% We also hope that this first-principles QED-DFT approach can inspire
% further exploration into fundamental beam-matter interactions to enable the
% modelling of a rich variety beam-induced phenomena.
